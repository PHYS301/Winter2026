% Options for packages loaded elsewhere
\PassOptionsToPackage{unicode}{hyperref}
\PassOptionsToPackage{hyphens}{url}
\PassOptionsToPackage{dvipsnames,svgnames,x11names}{xcolor}
%
\documentclass[
  letterpaper,
  DIV=11,
  numbers=noendperiod]{scrartcl}

\usepackage{amsmath,amssymb}
\usepackage{iftex}
\ifPDFTeX
  \usepackage[T1]{fontenc}
  \usepackage[utf8]{inputenc}
  \usepackage{textcomp} % provide euro and other symbols
\else % if luatex or xetex
  \usepackage{unicode-math}
  \defaultfontfeatures{Scale=MatchLowercase}
  \defaultfontfeatures[\rmfamily]{Ligatures=TeX,Scale=1}
\fi
\usepackage{lmodern}
\ifPDFTeX\else  
    % xetex/luatex font selection
\fi
% Use upquote if available, for straight quotes in verbatim environments
\IfFileExists{upquote.sty}{\usepackage{upquote}}{}
\IfFileExists{microtype.sty}{% use microtype if available
  \usepackage[]{microtype}
  \UseMicrotypeSet[protrusion]{basicmath} % disable protrusion for tt fonts
}{}
\makeatletter
\@ifundefined{KOMAClassName}{% if non-KOMA class
  \IfFileExists{parskip.sty}{%
    \usepackage{parskip}
  }{% else
    \setlength{\parindent}{0pt}
    \setlength{\parskip}{6pt plus 2pt minus 1pt}}
}{% if KOMA class
  \KOMAoptions{parskip=half}}
\makeatother
\usepackage{xcolor}
\setlength{\emergencystretch}{3em} % prevent overfull lines
\setcounter{secnumdepth}{5}
% Make \paragraph and \subparagraph free-standing
\ifx\paragraph\undefined\else
  \let\oldparagraph\paragraph
  \renewcommand{\paragraph}[1]{\oldparagraph{#1}\mbox{}}
\fi
\ifx\subparagraph\undefined\else
  \let\oldsubparagraph\subparagraph
  \renewcommand{\subparagraph}[1]{\oldsubparagraph{#1}\mbox{}}
\fi


\providecommand{\tightlist}{%
  \setlength{\itemsep}{0pt}\setlength{\parskip}{0pt}}\usepackage{longtable,booktabs,array}
\usepackage{calc} % for calculating minipage widths
% Correct order of tables after \paragraph or \subparagraph
\usepackage{etoolbox}
\makeatletter
\patchcmd\longtable{\par}{\if@noskipsec\mbox{}\fi\par}{}{}
\makeatother
% Allow footnotes in longtable head/foot
\IfFileExists{footnotehyper.sty}{\usepackage{footnotehyper}}{\usepackage{footnote}}
\makesavenoteenv{longtable}
\usepackage{graphicx}
\makeatletter
\def\maxwidth{\ifdim\Gin@nat@width>\linewidth\linewidth\else\Gin@nat@width\fi}
\def\maxheight{\ifdim\Gin@nat@height>\textheight\textheight\else\Gin@nat@height\fi}
\makeatother
% Scale images if necessary, so that they will not overflow the page
% margins by default, and it is still possible to overwrite the defaults
% using explicit options in \includegraphics[width, height, ...]{}
\setkeys{Gin}{width=\maxwidth,height=\maxheight,keepaspectratio}
% Set default figure placement to htbp
\makeatletter
\def\fps@figure{htbp}
\makeatother

\KOMAoption{captions}{tableheading}
\makeatletter
\@ifpackageloaded{caption}{}{\usepackage{caption}}
\AtBeginDocument{%
\ifdefined\contentsname
  \renewcommand*\contentsname{Table of contents}
\else
  \newcommand\contentsname{Table of contents}
\fi
\ifdefined\listfigurename
  \renewcommand*\listfigurename{List of Figures}
\else
  \newcommand\listfigurename{List of Figures}
\fi
\ifdefined\listtablename
  \renewcommand*\listtablename{List of Tables}
\else
  \newcommand\listtablename{List of Tables}
\fi
\ifdefined\figurename
  \renewcommand*\figurename{Figure}
\else
  \newcommand\figurename{Figure}
\fi
\ifdefined\tablename
  \renewcommand*\tablename{Table}
\else
  \newcommand\tablename{Table}
\fi
}
\@ifpackageloaded{float}{}{\usepackage{float}}
\floatstyle{ruled}
\@ifundefined{c@chapter}{\newfloat{codelisting}{h}{lop}}{\newfloat{codelisting}{h}{lop}[chapter]}
\floatname{codelisting}{Listing}
\newcommand*\listoflistings{\listof{codelisting}{List of Listings}}
\makeatother
\makeatletter
\makeatother
\makeatletter
\@ifpackageloaded{caption}{}{\usepackage{caption}}
\@ifpackageloaded{subcaption}{}{\usepackage{subcaption}}
\makeatother
\ifLuaTeX
  \usepackage{selnolig}  % disable illegal ligatures
\fi
\usepackage{bookmark}

\IfFileExists{xurl.sty}{\usepackage{xurl}}{} % add URL line breaks if available
\urlstyle{same} % disable monospaced font for URLs
\hypersetup{
  pdftitle={FAQ},
  colorlinks=true,
  linkcolor={blue},
  filecolor={Maroon},
  citecolor={Blue},
  urlcolor={Blue},
  pdfcreator={LaTeX via pandoc}}

\title{FAQ}
\author{}
\date{}

\begin{document}
\maketitle

\renewcommand*\contentsname{Table of contents}
{
\hypersetup{linkcolor=}
\setcounter{tocdepth}{3}
\tableofcontents
}
\section{Where do I get my textbook?}\label{where-do-i-get-my-textbook}

The textbook for this class is ````Mathematical Methods in Engineering
and Physics'' (Wiley, 2016) by Felder and Felder. The library should
have copies available, but you can also purchase it at the Bates
bookstore.

To find the library copies, go to \url{librarysearch.bates.edu} and
search for the course number or my name. Write down the book's call
number and ask for it at the library.

\includegraphics[width=2.08333in,height=\textheight]{images/FelderFelder.jpg}

\section{When and where is class?}\label{when-and-where-is-class}

These details are posted in Lyceum, and are in the google calendar event
sent to you by the registrar when you signed up for the class.

\section{What should I call you?}\label{what-should-i-call-you}

While my formal title would be ``Doctor'' or ``Professor,'' I don't
require you to use that title with me. I am more than comfortable being
referred to as ``Casey,'' so what you choose to call me will depend on
your own level of comfort.

I just ask that you never call me ``Ms.'' or ``Mrs.'' -- if you are
going to use a formal title for me, please use the one I earned over
five years of hard work in my doctorate. If you don't understand why
this distinction matters to me, I highly recommend that you look into
the
\href{https://www.liebertpub.com/doi/abs/10.1089/jwh.2016.6044?journalCode=jwh}{research}
or
\href{https://med.stanford.edu/news/all-news/2006/07/transgender-experience-led-stanford-scientist-to-critique-gender-difference.html}{anecdotal
evidence} on the disparities between how men and male-presenting people
are addressed versus how women and female-presenting people are
addressed, even when they have the same level of expertise.

\section{When are your office hours?}\label{officehours}

I will have three hours of office hours a week, plus time reserved in my
schedule for one-on-one meetings. You can schedule the meetings through
calendly (link in Lyceum).

The dates, times, and locations for office hours will be posted in
Lyceum. If you schedule a one-on-one meeting, it will be in my office.
The location of my office is posted in Lyceum, along with the calendly
link.

\section{Should I go to office hours?}\label{goingtoofficehours}

Office hours are my open hours, when I have specifically set aside time
to talk to my students (that's you!). Please always feel free to come
by. You can ask questions about homework, concepts from class, projects,
careers in physics, navigating the major, or anything else you want to
ask. I will prioritize questions about course content, but I'm always
happy to talk about other questions you might have. This is true of the
one-on-one meetings as well.

\section{What if I can't make it to your scheduled office
hours?}\label{what-if-i-cant-make-it-to-your-scheduled-office-hours}

It's impossible to find a time in my schedule that works in everyone
else's schedule, and sometimes it happens that I schedule office hours
that don't work for a few people. But this doesn't mean you can't get
help! If you can't make it to my scheduled office hours, you have two
options:

\begin{itemize}
\tightlist
\item
  Schedule a one-on-one meeting with me in calendly (link in Lyceum)
\item
  Go to SASC and work with a peer tutor
\end{itemize}

My calendly sometimes books up, especially close to exam dates, so I
recommend trying to schedule ahead of time. If you have an urgent
question and there are no available calendly slots, please feel free to
reach out to me via email.

\section{Can the syllabus change?}\label{can-the-syllabus-change}

Yes, under certain conditions. The schedule may change in small ways --
this will happen if there is a disruption in the schedule (e.g.~I am
sick and have to cancel class, a snow day prevents us from meeting, some
other emergency or surprise causes class to be cancelled). More often,
what will happen is we will need a little more time on a topic, and I
will push other topics a day or two. I build in a number of ``catch-up
days'' into the schedule in order to make sure that when this happens,
it doesn't require a total rewrite of the class.

The syllabus may also change if we agree together as a class to cut or
modify an assignment, or if I determine that the grading scheme has set
the bar too high. I will never modify assignments or grading in a way
that harms your grade, and if a change to either of those has you
concerned, please come talk to me and we will determine together a
reasonable alternative if that is needed.

Finally, I reserve the right always to modify typos, etc. If you spot
one of these, let me know, so I can fix it!

Any changes to the syllabus will be documented in the
\href{changelog.qmd}{changelog}

\section{Where do I find assignment instructions/course
handouts/slides?}\label{findstuff}

I created this website to make the syllabus and basic questions about
the course easy, but not everything will be here. Handouts, in-class
activities, specific assignments, and class slides will all be in the
course page on Lyceum.

If you are enrolled in the class but do not have access to Lyceum,
please let me know immediately so I can add you.

\section{When is \_\_ due?}\label{duedates}

You can find all due dates on the course schedule (link at the top of
the page). Dates are subject to change, but will always be kept up to
date on the schedule so check back if you're not sure.

\section{Can I have an extension?}\label{extensionpolicy}

In general, I am happy to work with you on deadlines, but it's important
for you to communicate with me. You can request extensions using
\href{https://forms.gle/zwnNC37WGg4fV3ZN8}{this form}. Note that some
assignments cannot be extended, and you can learn more about that by
reading more about the \href{syllabus.qmd\#assignments}{specific
assignments}. Requesting an extension on the form for something that
can't be extended will not result in an extension. If you have a special
case that you think qualifies as an exception (e.g.~an extended medical
absence, family emergency, etc), please email me to work out a new
schedule for assignments.

Please note that the 48 hour extension in the form is automatically
granted for extendable assignments, while anything longer than that
requires that we have a conversation. That conversation starts by you
proposing a new deadline in the form, and then I can talk to you about
what's possible or reasonable given the course schedule, your
circumstances, and other factors that may vary as the semester
progresses.

For more on my deadlines and extension policy, see the
\href{syllabus.qmd\#deadlines}{syllabus}.

\section{Can I re-do something I did poorly on?}\label{revisionpolicy}

Some assessments can be revised or re-tried. Please read up on each kind
of assignment in the class to see which ones can be revised or not.
Generally, if something is graded EMRN and you receive an R, you can
retry or revise it, provided there is enough time left in the semester
for a revision to occur (e.g.~if you turn something in to me on the last
day of the exam period, it cannot be revised due to time constraints).
You also cannot revise anything you don't turn in (receiving an N means
you lose the chance to try again, so turn things in!).

\section{When will you reply to my email?}\label{emails}

Timely communication is really important, but in order to be an
effective instructor and a well-adjusted human, I also need to be able
to take breaks from email, and I frequently get over a hundred emails in
a given day. If your email needs a reply, I will try to get back within
48 hours, but if it's timely and I haven't gotten back to you, please
feel free to follow up or speak to me in person. Please make sure you've
checked the syllabus, Lyceum, and your own emails for answers to
questions before you email me to ask about a due date or request an
extension.

\section{Do you grade on a curve?}\label{gradingcurve}

Grades are not curved; your grade depends only on your own performance.
Supporting your fellow students will help every one of you, and I hope
you will work together on homeworks and study together for tests.

\section{Can I use ChatGPT or Google Gemini in this
class?}\label{ChatGPT}

While I do not outright ban the use of LLMs, I do expect you to educate
yourself on their impacts (social, intellectual, environmental, and
economic), and to use them appropriately, cautiously, and with
integrity. In my \href{syllabus.qmd\#llms}{AI policy} I include links to
some articles that I hope you will see just as a starting point in
learning about these issues. If you encounter other articles that you
find informative, interesting, or useful about this topic, please pass
them on to me! I would love to read them.

Regarding the practical aspects of using LLMs in your work for this
course, the first and most important point is that you may never pass
off work done by an LLM as your own work if you wish to receive credit
for the assignment. Please make sure you read my full \hyperref[llms]{AI
policy}, including the articles I linked, and ask me if you have any
questions about the policy or are not sure if your use of an LLM would
fall under an acceptable use.

\section{What should I do if I get behind on my work?}\label{catchingup}

First, reach out to me. It helps me to know that you are working on
catching up. I am also happy to meet and help you set adjusted deadlines
to get back on track.

Second, if you are feeling really overwhelmed, consider scheduling an
appointment with
\href{https://www.bates.edu/student-academic-support-center/learning-strategies/}{a
learning strategies tutor} at the Student Academic Support Center. Their
role is to support students seeking help with time management,
organization, reading, test-taking, note-taking, and other academic
skills. They can help you talk through what strategies work for you,
what strategies don't, and how to manage your time and energy in a more
sustainable way.

\section{I have academic accommodations. What should I
do?}\label{accommodations}

I have tried to bring the concept of
\href{https://www.washington.edu/doit/what-universal-design-0}{universal
design} into how I have planned and structured this course, so I hope
that any accommodations you have are already built into this course
(aside from extra time on tests, which can't be accommodated during
class times, but you can take the tests through
\href{https://www.bates.edu/accessible-education-student-support/requesting-services/how-to-register-for-accommodations/}{Accessible
Education and Student Support}. However, I don't expect to have done
this perfectly (as there is no such thing), so if you have need of
certain accommodations that are not already provided by this class,
please let me know and I will do my best to meet those needs.

\section{What if I need more support in this class?}\label{extrahelp}

The Student Academic Support Center (SASC) has lots of tutors who can
help you strengthen your math skills, problem solving skills, and study
skills. Please reach out to them to see how they can help, in addition
to coming to my office hours, where I am more than happy to walk through
problems with you.



\end{document}
