% Options for packages loaded elsewhere
\PassOptionsToPackage{unicode}{hyperref}
\PassOptionsToPackage{hyphens}{url}
\PassOptionsToPackage{dvipsnames,svgnames,x11names}{xcolor}
%
\documentclass[
  letterpaper,
  DIV=11,
  numbers=noendperiod]{scrartcl}

\usepackage{amsmath,amssymb}
\usepackage{iftex}
\ifPDFTeX
  \usepackage[T1]{fontenc}
  \usepackage[utf8]{inputenc}
  \usepackage{textcomp} % provide euro and other symbols
\else % if luatex or xetex
  \usepackage{unicode-math}
  \defaultfontfeatures{Scale=MatchLowercase}
  \defaultfontfeatures[\rmfamily]{Ligatures=TeX,Scale=1}
\fi
\usepackage{lmodern}
\ifPDFTeX\else  
    % xetex/luatex font selection
\fi
% Use upquote if available, for straight quotes in verbatim environments
\IfFileExists{upquote.sty}{\usepackage{upquote}}{}
\IfFileExists{microtype.sty}{% use microtype if available
  \usepackage[]{microtype}
  \UseMicrotypeSet[protrusion]{basicmath} % disable protrusion for tt fonts
}{}
\makeatletter
\@ifundefined{KOMAClassName}{% if non-KOMA class
  \IfFileExists{parskip.sty}{%
    \usepackage{parskip}
  }{% else
    \setlength{\parindent}{0pt}
    \setlength{\parskip}{6pt plus 2pt minus 1pt}}
}{% if KOMA class
  \KOMAoptions{parskip=half}}
\makeatother
\usepackage{xcolor}
\setlength{\emergencystretch}{3em} % prevent overfull lines
\setcounter{secnumdepth}{5}
% Make \paragraph and \subparagraph free-standing
\ifx\paragraph\undefined\else
  \let\oldparagraph\paragraph
  \renewcommand{\paragraph}[1]{\oldparagraph{#1}\mbox{}}
\fi
\ifx\subparagraph\undefined\else
  \let\oldsubparagraph\subparagraph
  \renewcommand{\subparagraph}[1]{\oldsubparagraph{#1}\mbox{}}
\fi


\providecommand{\tightlist}{%
  \setlength{\itemsep}{0pt}\setlength{\parskip}{0pt}}\usepackage{longtable,booktabs,array}
\usepackage{calc} % for calculating minipage widths
% Correct order of tables after \paragraph or \subparagraph
\usepackage{etoolbox}
\makeatletter
\patchcmd\longtable{\par}{\if@noskipsec\mbox{}\fi\par}{}{}
\makeatother
% Allow footnotes in longtable head/foot
\IfFileExists{footnotehyper.sty}{\usepackage{footnotehyper}}{\usepackage{footnote}}
\makesavenoteenv{longtable}
\usepackage{graphicx}
\makeatletter
\def\maxwidth{\ifdim\Gin@nat@width>\linewidth\linewidth\else\Gin@nat@width\fi}
\def\maxheight{\ifdim\Gin@nat@height>\textheight\textheight\else\Gin@nat@height\fi}
\makeatother
% Scale images if necessary, so that they will not overflow the page
% margins by default, and it is still possible to overwrite the defaults
% using explicit options in \includegraphics[width, height, ...]{}
\setkeys{Gin}{width=\maxwidth,height=\maxheight,keepaspectratio}
% Set default figure placement to htbp
\makeatletter
\def\fps@figure{htbp}
\makeatother

\KOMAoption{captions}{tableheading}
\makeatletter
\@ifpackageloaded{caption}{}{\usepackage{caption}}
\AtBeginDocument{%
\ifdefined\contentsname
  \renewcommand*\contentsname{Table of contents}
\else
  \newcommand\contentsname{Table of contents}
\fi
\ifdefined\listfigurename
  \renewcommand*\listfigurename{List of Figures}
\else
  \newcommand\listfigurename{List of Figures}
\fi
\ifdefined\listtablename
  \renewcommand*\listtablename{List of Tables}
\else
  \newcommand\listtablename{List of Tables}
\fi
\ifdefined\figurename
  \renewcommand*\figurename{Figure}
\else
  \newcommand\figurename{Figure}
\fi
\ifdefined\tablename
  \renewcommand*\tablename{Table}
\else
  \newcommand\tablename{Table}
\fi
}
\@ifpackageloaded{float}{}{\usepackage{float}}
\floatstyle{ruled}
\@ifundefined{c@chapter}{\newfloat{codelisting}{h}{lop}}{\newfloat{codelisting}{h}{lop}[chapter]}
\floatname{codelisting}{Listing}
\newcommand*\listoflistings{\listof{codelisting}{List of Listings}}
\makeatother
\makeatletter
\makeatother
\makeatletter
\@ifpackageloaded{caption}{}{\usepackage{caption}}
\@ifpackageloaded{subcaption}{}{\usepackage{subcaption}}
\makeatother
\ifLuaTeX
  \usepackage{selnolig}  % disable illegal ligatures
\fi
\usepackage{bookmark}

\IfFileExists{xurl.sty}{\usepackage{xurl}}{} % add URL line breaks if available
\urlstyle{same} % disable monospaced font for URLs
\hypersetup{
  pdftitle={Syllabus},
  colorlinks=true,
  linkcolor={blue},
  filecolor={Maroon},
  citecolor={Blue},
  urlcolor={Blue},
  pdfcreator={LaTeX via pandoc}}

\title{Syllabus}
\author{}
\date{}

\begin{document}
\maketitle

\renewcommand*\contentsname{Table of contents}
{
\hypersetup{linkcolor=}
\setcounter{tocdepth}{3}
\tableofcontents
}
\section{Who is the professor?}\label{who-is-the-professor}

I'm Casey Berger and this is my second year at Bates College. At Bates,
I've taught PHYS 211, PHYS 216, PHYS 301, PHYS 308, and PHYS 409. Before
Bates, I was at Smith College in Physics and Statistical and Data
Sciences. And long before that -- before I went to school for physics
and computational science -- I studied philosophy, film production, and
Spanish and worked in the film industry. It took me a long time to find
my way to physics, but I wouldn't change any of it. The long and winding
path taught me so much more about who I am, what I am capable of, and
how many opportunities are available in the world. It also taught me
that it's never too late to find a new passion or change your career
direction.

My research involves applying high performance computing and data
science techniques to many-body quantum systems, which just means
studying how medium-to-large numbers of quantum objects like electrons,
neutrons, or atoms interact with each other and their environment. This
sounds straightforward, but unfortunately the mathematics of quantum
mechanics means these problems become impossible to solve by hand and
extremely challenging to solve even with high powered computers when the
system is still only a few particles.

Outside of physics, I am a person with lots of interests and hobbies. I
love being outdoors in all seasons, but also love reading a book inside
with a cup of tea. Ask me for book recommendations, recipes for
interesting food, or great hikes in Maine!

\section{Classroom expectations}\label{classexpectations}

\subsection{What you can expect from me}

\begin{itemize}
\tightlist
\item
  I will stay home if I am feeling sick and make arrangements to deliver
  the course material
\item
  I will work with you to arrange accommodations when you need them
\item
  I will respect your time by starting and ending class on time
\item
  I will answer your questions thoughtfully, and if I don't know the
  answer, I will follow up in a timely manner
\item
  I will embrace who you are as whole people
\item
  I will model respect, openness, and engagement, and foster a
  supportive and inclusive environment
\item
  I will be honest when I make mistakes, because failure is part of
  growing
\end{itemize}

\subsection{What I expect from you}

\begin{itemize}
\tightlist
\item
  That you will stay home if you are sick and contact me via email to
  arrange accommodations
\item
  That you genuinely attempt to engage with the course
\item
  That you ask questions if you are confused (you may do this privately
  -- there is no obligation to ask during class hours)
\item
  That you communicate with me when you have problems that interfere
  with your ability to engage with the coursework
\item
  That you treat your peers with respect and openness, and that you
  participate in creating an inclusive, supportive, and engaged
  classroom
\end{itemize}

\subsection{What is not expected}

\begin{itemize}
\tightlist
\item
  Perfection. Ever. It's a myth.
\item
  That you will `sit still' or ask for permission to leave the classroom
  to go to the bathroom or if you just need a minute.
\item
  That everyone will learn in the same way. You do not have to match
  some ``model student'' to do well in this class
\end{itemize}

\section{Team-Based Learning}\label{tbl}

I am using a variation of team-based learning for this class, in order
to cultivate a community-minded classroom, encourage a growth mindset,
and build group work skills. Here is how this will work:

We will have three modules, and you will work in a team of 4-5 students
for each module. Your team will work on problems in class, discuss the
content, and turn in a weekly problem that you solve together. At the
end of each module, you will provide feedback on your teammates and on
your own work, and then we will shuffle the groups, so you will have
three different teams over the course of the semester.

In order to be a good member of your team, there will be a reading quiz
and short warm-up problem (the PCQ) that must be completed prior to
class. If you do not complete it prior to class, you must work alone to
finish the assignment before you can join your team. I will take into
account your attendance and whether you did the PCQs prior to class when
assessing your community engagement for the module.

Not all of the work will be in groups. You will also have individual
quizzes and homework.

I have tried to balance the class so that there is a mix of individual
and group work, which I hope will allow everyone to get something
meaningful out of the course.

\section{Deadlines and Extensions}\label{deadlines}

If you need an extension, you may request one using
\href{https://forms.gle/eFx7y7FoSdoukKGC6}{this form}. I recognize that
things come up and you may require flexibility at some point in the
semester. Please feel free to reach out to me directly if you are
struggling to meet a deadline. I want to support you and make sure you
have the best possible chance for success in this class, and the only
way I can help is if you communicate with me. Extension requests are
always due before the deadline. Work submitted after the deadline
without an approved extension will not be graded.

In general, I am happy to be flexible. Please note, however, that some
assignments will have stricter deadlines, which are discussed in their
descriptions above.

\section{Technology Policy}\label{techpolicy}

\subsection{Use of LLMs}\label{llms}

Large Language Models (LLMs) are here to stay, but we should be
discerning about how we use them. Banning them from the classroom is
both pointless and actively unhelpful. If you are going to use ChatGPT
or a similar LLM in this class, I only require two things:

\begin{enumerate}
\def\labelenumi{\arabic{enumi}.}
\tightlist
\item
  That you do a little reading on Large Language Models. Specifically,
  read
\end{enumerate}

\begin{itemize}
\item
  \href{https://stackoverflow.blog/2023/07/03/do-large-language-models-know-what-they-are-talking-about/}{this
  article about how LLMs actually work}
\item
  \href{https://montrealethics.ai/what-lies-behind-agi-ethical-concerns-related-to-llms/}{this
  article about the ethical considerations around LLMs and intellectual
  property}
\item
  \href{https://heated.world/p/data-centers-arent-just-guzzling}{this
  article about just one of the many environmental impacts of LLMs}
\end{itemize}

\begin{enumerate}
\def\labelenumi{\arabic{enumi}.}
\setcounter{enumi}{1}
\tightlist
\item
  That you consider the LLM a source that you must cite. If you use
  generative AI to help with any assignment, you \textbf{must}
  acknowledge that help and give a short (two-sentence) description of
  what the LLM did for you and how it helped you solve the problem.
\end{enumerate}

Remember that LLMs are designed to predict the most probable response to
a question, which is not always the best or even correct response. They
are prone to ``hallucinating'' (making stuff up, often in a way that is
convincing but still false). They can't do creative problem solving,
which is the most important skill we are trying to develop in this
class.

\subsection{Use of technology during class time}\label{distractions}

Technology is completely infused into our lives, and I'm not trying to
convince anyone to stop using it. As a computational physicist, it's my
literal vocation to use digital technology, and as a teacher, I spend
many hours a day staring at my screen to write lesson plans and respond
to emails. But it's important to know that as useful a tool as
technology is, it can also inhibit us.

First, the science. While we all believe ourselves to be excellent
multitaskers (okay maybe that's hyperbole), studies show that we can't
actually focus on more than one thing at a time - what we are doing is
actually rapidly task switching between one task and the next, and that
\href{https://www.apa.org/topics/research/multitasking}{task switching
takes time and energy, making us do both tasks more poorly}. This means
that if you're toggling between your email, social media, text messages,
or whatever other content might be up on your laptop, phone, or tablet
in class, you are going to have a harder time learning the content in
class \emph{and} a harder time doing whatever else you are trying to do.
In addition, studies have found that
\href{https://www.journals.uchicago.edu/doi/full/10.1086/691462}{merely
being in the presence of our smartphones can cause us to be distracted
by them}.

I have found this to be extremely true, and as a general rule, if I'm
working on something that requires my focus, such as writing a lesson
plan or working on research projects, I have a box in my office that I
put my phone in so I can't be distracted by it.

Our goal in this class is to learn, and class time is an important time
set aside for us to meet together in community to work towards the
learning goals. In order to make sure you get the most out of that time,
and to help support your classmates to not be distracted, I ask that you
do the following:

\begin{itemize}
\tightlist
\item
  Keep your phone on silent and out of view throughout the duration of
  class
\item
  Close all windows on your laptop or tablet that are not either your
  notes for the class, the Lyceum page or website for the class, a
  Mathematica notebook, or a PDF of the textbook
\end{itemize}

If the second one feels intimidating (what if you lose a window that was
important for something you need to do later?), look into browser
extensions that allow you to save tabs for later, like
\href{https://addons.mozilla.org/en-US/firefox/addon/tab-session-manager/}{Mozilla
Session Manager for Firefox} or
\href{https://chromewebstore.google.com/detail/onetab/chphlpgkkbolifaimnlloiipkdnihall?utm_source=chrome-app-launcher-search}{OneTab
for Chrome}.

All that said, I recognize that sometimes we have situations that
require us to be reachable, so if you have a special circumstance that
means you need to be able to take a call when it comes in, please talk
to me about an exception to this policy.

Since this policy is intended to support all members of the community, I
will ask that we all work together to encourage each other to remain
present during our class time. To do this, here is how we will put this
policy into practice:

\begin{itemize}
\tightlist
\item
  If you find yourself distracted by something on your computer, please

  \begin{enumerate}
  \def\labelenumi{\arabic{enumi}.}
  \tightlist
  \item
    acknowledge it
  \item
    apologize if this occurred during group work (this doesn't have to
    be a drawn out apology, just ``I'm sorry, I got distracted, but I'm
    back now'')
  \item
    remove the distraction and return your attention to class.
  \end{enumerate}
\item
  If you notice your classmate is distracted, please

  \begin{enumerate}
  \def\labelenumi{\arabic{enumi}.}
  \tightlist
  \item
    acknowledge what you see gently (``Hey, I notice you're checking
    your email right now'')
  \item
    thank them for returning their attention to the task at hand
  \item
    move ahead with what you were doing (don't make it a big event)
  \end{enumerate}
\item
  If I notice someone is distracted once (including having tabs open
  that are not relevant to the class), I will

  \begin{enumerate}
  \def\labelenumi{\arabic{enumi}.}
  \tightlist
  \item
    acknowledge it as discreetly as I can
  \item
    ask you to close the distraction
  \end{enumerate}
\item
  If I notice someone is repeatedly distracted by the same thing during
  one class session, I will

  \begin{enumerate}
  \def\labelenumi{\arabic{enumi}.}
  \tightlist
  \item
    note the previous instances
  \item
    ask you to put away whatever device the distraction was on. If this
    means you need paper to take notes on, I will provide that.
  \end{enumerate}
\item
  If the distraction recurs over the course of the semester, it will
  affect your community engagement grade
\end{itemize}

\section{Assignments and Grading}\label{grading}

Assignments fall into ``bundles,'' which contribute to your grade
equally. Your performance on each bundle determines your base grade (by
averaging the grade in each bundle using the 4-point GPA scale). Beyond
that, you can achieve grade boosts, which round your grade up, e.g.~from
a B to a B+, or a B+ to an A-.

You can learn more about each bundle below:

\hyperref[practice]{Practice}

\hyperref[groupwork]{Group Work}

\hyperref[quizzes]{Quizzes}

\hyperref[excellence]{Excellence}

\subsection{Grading Scheme}\label{gradingscheme}

At the end of the term, you will earn a letter grade for this course.
That letter grade will be determined following the Four-Square grading
chart (below). Each square represents a grading bundle, and you can earn
a score of 0-4 for each bundle. At the end of the term, I will calculate
an average score for you using the following equation:

Overall Score = (Practice score + Group Work score + Quizzes score +
Excellence score) / 16

\includegraphics[width=10.41667in,height=\textheight]{images/PHYS301foursquare.png}

The resulting overall score will correspond to a letter grade using the
percentage table below:

\begin{longtable}[]{@{}ll@{}}
\caption{Grade table}\tabularnewline
\toprule\noalign{}
Letter Grade & Percentage \\
\midrule\noalign{}
\endfirsthead
\toprule\noalign{}
Letter Grade & Percentage \\
\midrule\noalign{}
\endhead
\bottomrule\noalign{}
\endlastfoot
A+ & 97-100\% \\
A & 93-96.9\% \\
A- & 90-92.9\% \\
B+ & 87-89.9\% \\
B & 83-86.9\% \\
B- & 80-82.9\% \\
C+ & 77-79.9\% \\
C & 73-76.9\% \\
C- & 70-72.9\% \\
D+ & 67-69.9\% \\
D & 63-66.9\% \\
D- & 60-62.9\% \\
F & less than 60\% \\
\end{longtable}

You can then get up to three grade boosts, each of which rounds your
grade up (e.g from a B to B+ or from an A- to A). The
\hyperref[boosts]{grade boosts} are described more below.

If, for example, you achieved a 3 in Practice, a 4 in Group Work, a 3 in
Quizzes, and got a total of 16 ``Es'', your base grade would be (3.0 +
4.0 + 3.0 + 3.0)/16 = = 81.3\%, which would be a B-. With round-ups,
your grade could go up to a B, B+, or A-.

\subsection{Grading scales}\label{grading-scales}

\subsubsection{Points and Completion}

Homeworks will be graded on a points scale, and PCQs on whether you
completed them (regardless of ``correctness'').

\subsubsection{E/M/R/N}

Quizzes, group problems, and community engagement are graded on this
scale

\begin{itemize}
\tightlist
\item
  E: exceeds expectations
\item
  M: meets expectations
\item
  R: revise / retry
\item
  N: no submission
\end{itemize}

E and M are considered passing grades. The number of Es you get will
contribute to your overall grade (see the
\hyperref[gradingscheme]{grading scheme})

R and N are considered failing grades.

In the case of quizzes and group problems, you will have one chance to
revise/retry (see more under the assignment descriptions). Community
engagement cannot be revised, as it is a reflection of your
participation and engagement throughout the unit.

Under no circumstances can you revise something you receive an N on.
Remember that done is better than perfect -- turn your work in so you
can receive feedback.

\subsection{Grading bundles
(\#assignments)}\label{grading-bundles-assignments}

\subsubsection{Practice}

PCQs: In order to get the most out of class and be a good team member,
you need to prepare. To do this, we will have daily reading and
pre-class questions, which are google forms that ask you a few questions
about the reading and ask you to work through a warm-up problem, which
you must submit prior to class start. All I ask is that you show a
good-faith effort to get full credit. You do not have to solve the
warm-up problem correctly, and you will go over it with your group in
class.

\emph{Given the purpose of the PCQs, I will not offer extensions or
revisions, except in case of a major medical or family emergency that
causes you to miss class.}

WHWs: Regular problem-solving is very important to developing confidence
and skill in this content, so every week you will have a short weekly
problem set, which will be graded by a grader.

You may request extensions on the WHWs using the
\href{https://forms.gle/eFx7y7FoSdoukKGC6}{extension form}.

\begin{longtable}[]{@{}
  >{\raggedright\arraybackslash}p{(\columnwidth - 2\tabcolsep) * \real{0.2308}}
  >{\raggedright\arraybackslash}p{(\columnwidth - 2\tabcolsep) * \real{0.7692}}@{}}
\caption{Bundle grades}\tabularnewline
\toprule\noalign{}
\begin{minipage}[b]{\linewidth}\raggedright
Points
\end{minipage} & \begin{minipage}[b]{\linewidth}\raggedright
Requirements
\end{minipage} \\
\midrule\noalign{}
\endfirsthead
\toprule\noalign{}
\begin{minipage}[b]{\linewidth}\raggedright
Points
\end{minipage} & \begin{minipage}[b]{\linewidth}\raggedright
Requirements
\end{minipage} \\
\midrule\noalign{}
\endhead
\bottomrule\noalign{}
\endlastfoot
4.0 & Earn \textgreater= 90\% average on WHWs AND complete \textgreater=
90\% of PCQs \\
3.0 & Earn \textgreater= 80\% average on WHWs AND complete \textgreater=
80\% of PCQs \\
2.0 & Earn \textgreater= 70\% average on WHWs AND complete \textgreater=
70\% of PCQs \\
1.0 & Earn \textgreater= 60\% average on WHWs AND complete \textgreater=
60\% of PCQs \\
0.0 & Under a 60\% average on WHWs OR under a 60\% average on PCQs \\
\end{longtable}

\subsubsection{Group Work}

Group Problems: Each week, your team will be responsible for completing
and doing a formal write-up of one problem. You will submit one write-up
together on Lyceum, which must be type-set and clearly organized. You
may type-set using LaTeX/overleaf, Google docs, or any other word
processing software that can handle mathematical expressions. For each
part, make sure you include at least one sentence that describes the
reasoning/explanation for the math you applied. This will be graded on
correctness, thoroughness, and clarity, and will receive a grade of E,
M, R, or N. Group problems receiving an R may be revised once.

You may request an extension as a group on these problems using the
\href{https://forms.gle/eFx7y7FoSdoukKGC6}{extension form}.

Community Engagement: At the end of each module, you and your peers will
fill out a form where you provide feedback on your teammates'
contributions and your own. Your self-assessment and your peers'
assessments will be weighted equally in the grade, and I will also take
into account your participation in pre-class quizzes and your in-class
attendance in order to determine a final grade for each module.

These are graded on an \hyperref[emrn]{E/M/R/N scale}. Not submitting
your own form will result in an ``N'' regardless of how your team rates
your participation.

I recognize that group work and peer grading can be fraught. I have been
in many difficult group projects myself, and as a queer woman in
physics, I'm very aware of cultural and social biases that exist in our
field. If there are any concerning dynamics in your group, I hope that
you feel safe talking to me in addition to including that information in
your feedback form. You may also fill out my
\href{https://forms.gle/h1dzxVJKPEVo3v467}{anonymous feedback form} and
give me only the information you think I need to address this issue
anonymously.

\begin{longtable}[]{@{}
  >{\raggedright\arraybackslash}p{(\columnwidth - 2\tabcolsep) * \real{0.2308}}
  >{\raggedright\arraybackslash}p{(\columnwidth - 2\tabcolsep) * \real{0.7692}}@{}}
\caption{Bundle grades}\tabularnewline
\toprule\noalign{}
\begin{minipage}[b]{\linewidth}\raggedright
Points
\end{minipage} & \begin{minipage}[b]{\linewidth}\raggedright
Requirements
\end{minipage} \\
\midrule\noalign{}
\endfirsthead
\toprule\noalign{}
\begin{minipage}[b]{\linewidth}\raggedright
Points
\end{minipage} & \begin{minipage}[b]{\linewidth}\raggedright
Requirements
\end{minipage} \\
\midrule\noalign{}
\endhead
\bottomrule\noalign{}
\endlastfoot
4.0 & At least M on 12 Group Problems. AND at least M on all Community
Engagements \\
3.0 & At least M on 10 Group Problems. AND at least M on all Community
Engagements \\
2.0 & At least M on 8 Group Problems. AND at least M on 2 Community
Engagements \\
1.0 & At least M on 6 Group Problems. AND at least M on 1 Community
Engagement \\
0.0 & Fewer than 6 Ms or higher on Group Problems OR no Ms or higher on
Community Engagement \\
\end{longtable}

\subsubsection{Weekly quizzes}

There will be a closed-book quiz each week on Friday. These will mostly
focus on recent content, but may sometimes require that you use skills
from previous units.

If you have to miss class or leave early on a Friday due to an approved
excuse, you may take the quiz at the testing center. You must do this no
more than 2 days before the quiz is given and no more than 5 days after
(i.e.~starting the Wednesday before the quiz and ending the Wednesday
after the quiz), and you can request this option through AESS using this
\href{https://www.bates.edu/accessible-education-student-support/makeup-exam-requests/}{form}.
If you are not sure your absence is excused or you feel you have a
special circumstance, come and talk to me about it and we will discuss
the options.

Quizzes are graded on an \hyperref[emrn]{E/M/R/N scale}. Each quiz will
have a clear rubric of which learning objectives are being assessed and
how to achieve an M or E on the quiz. There will be two opportunities to
retake quizzes: once in the middle of the semester, and once during the
``final exam'' time slot (there is no final for this class).

\begin{longtable}[]{@{}ll@{}}
\caption{Bundle grades}\tabularnewline
\toprule\noalign{}
Points & Requirements \\
\midrule\noalign{}
\endfirsthead
\toprule\noalign{}
Points & Requirements \\
\midrule\noalign{}
\endhead
\bottomrule\noalign{}
\endlastfoot
4.0 & At least M on all 10 \\
3.0 & At least M on 8 \\
2.0 & At least M on 6 \\
1.0 & At least M on 4 \\
0.0 & Fewer than 4 quizzes with M or higher \\
\end{longtable}

\subsubsection{Excellence}

There are 25 potential ``E'' grades you can earn in this class: 10
quizzes, 12 group problems, and 3 community engagement grades. Every
``E'' you receive will be counted in this category and goes toward your
final grade.

\begin{longtable}[]{@{}ll@{}}
\caption{Bundle grades}\tabularnewline
\toprule\noalign{}
Points & Requirements \\
\midrule\noalign{}
\endfirsthead
\toprule\noalign{}
Points & Requirements \\
\midrule\noalign{}
\endhead
\bottomrule\noalign{}
\endlastfoot
4.0 & At least 20 Es \\
3.0 & At least 15 Es \\
2.0 & At least 10 Es \\
1.0 & At least 5 Es \\
0.0 & Fewer than 5 Es \\
\end{longtable}

\subsection{Grading boosts}\label{boosts}

Your base grade will be calculated from the table above, but you can
round your grade up in a number of ways.

I will apply as many grade boosts as you achieve, with one exception:

You can only achieve an A+ in this class if you receive a base A grade.
The A+ grade does not have an impact on your GPA, and so I am reserving
it as a way of acknowledging especially excellent work across the course
of the semester.

\subsubsection{Time Management}

If you use no more than 3 extension requests all semester (on individual
assignments), I will round your grade up.

\subsubsection{Growth}

If you demonstrate consistent improvement in your work across the
semester, I reserve the right to round your grade up.

\subsubsection{Metacognition}

In order to get this boost, you must do 5 of the following 7
assignments:

\begin{itemize}
\tightlist
\item
  Read an article on growth mindset and reflect on how it applies to
  your own life
\item
  Fill out the first week of class form before the end of January
\item
  Set a SMART goal (this will be evaluated on whether it meets the
  criteria) for a study skill to improve. You must do this before the
  end of January.
\item
  Fill out the mid-semester survey before the end of February Recess
\item
  Schedule an appointment with
  \href{https://www.bates.edu/student-academic-support-center/learning-strategies/}{a
  learning strategies tutor} at SASC and write a reflection on one thing
  you learned at that meeting
\item
  Bring questions to SASC at least once and write a reflection about
  your experience
\item
  Come to my office hours at least once and write a reflection about
  your experience
\item
  Reflect on your SMART goal at the end of the semester
\end{itemize}

These assignments will all be posted in Lyceum under the
``Metacognition'' tab.



\end{document}
